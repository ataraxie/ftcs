\documentclass[sigconf]{acmart}
\usepackage[utf8]{inputenc}

\settopmatter{printacmref=false}
\renewcommand\footnotetextcopyrightpermission[1]{} % removes footnote with conference information in first column
\pagestyle{plain}

\usepackage{url}
\usepackage{hyperref}
\hypersetup{
    colorlinks=true,
    linkcolor=blue,
    filecolor=magenta,      
    urlcolor=cyan,
}
\urlstyle{same}


\usepackage{listings}

% Enabling Javascript syntax highlight in code snippet - BEGIN 
% https://tex.stackexchange.com/questions/89574/language-option-supported-in-listings
\usepackage{color}
\definecolor{lightgray}{rgb}{.9,.9,.9}
\definecolor{darkgray}{rgb}{.4,.4,.4}
\definecolor{purple}{rgb}{0.65, 0.12, 0.82}
\definecolor{darkgreen}{rgb}{0, .64, 0}

\lstdefinelanguage{JavaScript}{
  keywords={typeof, new, true, false, catch, function, return, null, catch, switch, var, if, in, while, do, else, case, break},
  keywordstyle=\color{blue}\bfseries,
  ndkeywords={class, export, boolean, throw, implements, import, this},
  ndkeywordstyle=\color{darkgray}\bfseries,
  identifierstyle=\color{black},
  sensitive=false,
  comment=[l]{//},
  morecomment=[s]{/*}{*/},
  commentstyle=\color{purple}\ttfamily,
  stringstyle=\color{darkgreen}\ttfamily,
  morestring=[b]',
  morestring=[b]"
}

\lstset{
   language=JavaScript,
   extendedchars=true,
   basicstyle=\footnotesize\ttfamily,
   showstringspaces=false,
   showspaces=false,
   numbers=left,
   numberstyle=\footnotesize,
   numbersep=9pt,
   tabsize=2,
   breaklines=true,
   showtabs=false,
   captionpos=b
}
% Enabling Javascript syntax highlight in code snippet - END

\newcommand{\APIName}{Successorships}

\newcommand{\APIshort}{ships}

%opening
\title{Who is the next server? Enabling fault-tolerant local Webapps}

\author{Arthur Marques \qquad Felix Grund \qquad Paul Cernek}
\affiliation{
    \institution{University of British Columbia}
    \city{Vancouver} 
    \state{BC} 
  }

\begin{document}

\maketitle

% \begin{abstract}

% \end{abstract}

\section{Introduction}
\label{sec:introduction}

\section{Background}
\label{sec:background}

In this section, we detail the underlying concepts that are related to our project proposal.

\subsection{Zero-configuration Networks}
\label{sec:zeroconf}



Zero-configuration networking is a combination of protocols that aims to automatically discover computers or peripherals in a network without any central servers or human administration. To do so, Zero-configuration networks have two major components that provide {\it (i)} the automatic assignment of IP addresses and host naming (mDNS), and {\it (ii)} service discovery (DNS-SD).


Roughly, when a device enters the local network, he assigns and IP/name to himself and then multicast that to the local network, resolving any name conflict that may occur in such process. IP assignment considers the link-local domain address, which draws addresses from the IPv4 169.254/16 prefix and, once an IP is selected, a host name with the suffix ``.local'' is mapped to that IP~\cite{rfc6762}. As devices are mapped to IPs/host names, their available services are discovered using a a combination of DNS PTR, SRV, and TXT records~\cite{rfc6763}, thus their services can be requested by other devices. 




\subsection{Local-area frameworks}
\label{sec:flyweb}

FlyWeb is a Web API developed by the Mozilla Firefox community which enables clients of Web applications to publish a local server from within the browser. Building on the concept of zero-configuration networks and its mDNS/DNS-SD protocols~\cite{rfc6762, rfc6763}, the server advertises itself in the local network and can be discovered by other devices which become clients to the server by connecting via a HTTP or WebSocket connection. This essentially enables cross-device communication within a local-area network.

Google Nearby Connections\footnote{\url{https://developers.google.com/nearby/connections/overview}} is a peer-to-peer networking API that allows apps to discover, connect and exchange data to other devices in a local area network. The Nearby Connections app work through a combination of Bluetooth, BLE, and Wifi hotspots in order to exchange data. However, the API does not provide details about protocols and how devices are resolved.

Opera Unite\footnote{\url{http://www.operasoftware.com/press/releases/general/opera-unite-reinvents-the-web}} also has a technology that starts a local server in the Opera browser. Their technology allows file sharing, photo sharing, and other features but, similar to Google Nearby Connections, their documentation does not clearly specify which technologies are used.

\subsection{Replication}
\label{sec:replication}

Fault tolerance and reliability in distributed systems with client-server architecture are generally achieved by data replication: information is shared on redundant server replicas such that any replica can become the new master if the current master fails. While improving system artifacts like fault-tolerance, reliability and availability, replication can come at the cost of performance: depending on the required operations in the system for replication, system performance can suffer significant bottlenecks. Different models of replication have been proposed to trade consistency for performance, resulting in different levels of consistency as a design choice for the target system. Traditionally, two strategies of replication are distinguished: \textit{active replication} and \textit{passive replication}. A third type of replication, \textit{lazy replication}, was later introduced and is gaining more attention recently. The following paragraphs describe these three types of replication.

\textbf{Active replication.} The first strategy (also called \textit{primary-backup} or \textit{master-slave}), requests to the master replica are processed to all other replicas. Given the same initial state and request sequence, all replicas will produce the same response sequence and reach the same final state. Active replication has become most prominent with the introduction of the State Machine Replication model which was introduced in the 1980s \cite{Lamport:1984} and later refined in \cite{Schneider:1990}. It is based on the concept of distributed consensus with the goal of reliably reaching a stable state of the system in the presence of failures. While providing small recovery delay after failures due to an imposed total order of state updates, computation performance can suffer tremendous bottlenecks since updates must be sequentially propagated through all replicas. 

\textbf{Passive replication.} The second strategy (also called \textit{multi-primary} or \textit{multi-master} scheme) relaxes sequential ordering: clients communicate with a master replica and updates are forwarded to backup replicas. Computation performance is improved with this pattern since all computation takes place on the master replica and only the results are propagated. The downside of the approach is that more network bandwidth is required if updates are large. Since the primary replica represents a single point of entry to clients with this approach, there must be some kind of distributed concurrency control in order to reliably restore state when the primary fails. This makes the implementation of this approach more complex and recovery potentially slower.

\textbf{Lazy replication.} A third strategy of replication was proposed in 1990: \textit{lazy replication} \cite{Ladin:1990,Ladin:1992} (also called \textit{optimistic replication}) aims at providing highest possible performance and availability by sacrificing consistency significantly. With this approach, replicas periodically exchange information, tolerating out-of-sync periods but guarantee to catch up eventually. While the traditional approaches guarantee from the beginning that all replicas have the exact same state at any point in time, lazy replication allows states to diverge on replicas, but guarantees that the states converge when the system quiesces for some time period. In contrast to the strong consistency models used in the traditional approaches, lazy replication is based on eventual consistency which has gained more attention recently, in particular in online editing platforms, NoSQL cloud databases and big data \footnote{http://www.oracle.com/technetwork/consistency-explained-1659908.pdf, accessed 2017-10-08}. Eventual consistency is the weakest consistency model, providing no guarantee for safety as long as replicas have not converged. Rather, it "push[es] the boundaries of highly available systems" \cite{Bailis:2013}. The introduction of \textit{conflict-free replicated data types} \cite{Shapiro:2011} aimed at a stronger model of eventual consistency: any two replicas that receive the same updates, no matter the order, will be in the same state. CFDTs are categorized in operation-based (only update operation is propagated) and state-based (full state is propagated). A number of CFDTs have been suggested, among them are sets, maps and graphs. It is important to mention that all eventual consistency models impact the application designer since she has to determine what level consistency is sufficient for the specific application.





% TODO (PTC): The broader term for what we want to implement is "failover", except in a distributed setting. Look into existing literature on this term.

\section{Proposed Approach}
\label{sec:approach}

Our goal is to build a framework to facilitate the development of offline client-server Web applications that robustly recover from server faults. 
We posit that the following features are prerequisites to achieving this:

\begin{enumerate}
    \item Nodes in the network trust each other;
    \item Any client, but exactly one client, has the ability to automatically assume the responsibilities of the server if the server goes down;
    \item All clients in the network have the capability to automatically update their connections to the new server in the event that the server migrates from one node to another;
    \item Communication in the network is not traffic intensive, i.e. nodes neither produce bursts of requests in a small period of time nor have payloads higher than some threshold $\tau$ that might generate bottlenecks in the network;
    
\end{enumerate}

The model we propose for achieving this is one in which clients connecting to the server automatically acquire distributed state including the following elements:

\begin{enumerate}
	\item Constant: A UUID for the initial host node (the first to serve the application);
    \item The current state of the server;
	\item A ``successorship'' list: a list of (potentially not all of the) nodes in the local network, in order of ``who is next'' to assume server responsibilities, in the event that the server goes down;
    \item Constant: The actual server code to execute, in the event that one of the clients needs to begin acting as the server.
\end{enumerate}

Note that the elements marked ``(Constant)'' are permanently fixed (for the lifetime of the application) when the initial server node first runs the application server.

We propose to develop a Javascript library that implements the functionality listed above, providing a clean interface to enable developers to seamlessly integrate fault-tolerance into their offline client-server Web applications, without having to worry about the details of how such fault-tolerance is achieved.

\subsection{API Overview}

Our current running name for the library is \texttt{\APIName}\footnote{We still need a proper acronym for it, e.g. \APIshort}, i.e. the next ship that will lead the flotilla after yet another sunk boat. We propose to implement the following interface in \APIName:


\begin{itemize}
	\item Server side:
    \begin{itemize}
    \begin{ttfamily}
      \item initServer(name)
      \item onReceive(msg, callback)
      \item commitState(callback)
    \end{ttfamily}
    \end{itemize}
    \item Client side:
    	\begin{itemize}
    	\begin{ttfamily}
    		\item initClient()
            \item connect(serverName)
            \item send(msg, payload)
    	\end{ttfamily}
    	\end{itemize}
\end{itemize}


We briefly discuss the major functions of our proposed API in the following subsections. Throughout the discussion, we use the TA queue example to illustrate our API usage.

{\bf Server initialization and service instantiation: } the first functions to initiate a server are {\ttfamily initServer} and {\ttfamily onReceive}. The former starts the local server in the device's browser and, after initialization, assigns a host name for that server. The later register entry points for services offered by that server.

In our queue system, one would initialize a server and define two functions to handle requests to enqueue students and also to dequeue them once they are helped. Additionally, the server provides the queue service in order to provide the current state of the queue. If no recognizable service is requested, the TAQueue server responds with the queue service.

\begin{lstlisting}[language=JavaScript]
    function getQueue(req, event) { ... }
    function handleEnqueue(req, event) { ... }
    function handleDequeue(req, event) { ... }

    (function main(){
        server = ship.initServer("TAQueue");
        server.onReceive('queue', getQueue, 
            default=true);
        server.onReceive('enqueue', handleEnqueue);
        server.onReceive('dequeue', handleDequeue);
    })();
\end{lstlisting}

{\bf Establishing connections: } as a server starts running, it broadcasts its name in the local-area network and clients in the same network can discover this server. A client device needs a single line of code to initialize itself. Upon initialization it will lookup for host servers in the network. Once a list of servers is retrieved and displayed, a client may select a server to connect to. In our TA queue example, we explicitly know the server name and skip the server list phase:

\begin{lstlisting}[language=JavaScript]
    (function main(){
        client = ship.initClient().connect("TAQueue");
    })();
\end{lstlisting}


{\bf Data exchange: } clients can ask for services through the {\ttfamily send} function. The function explicitly takes a requested service as one of its parameters and a payload as its second one. In our queue system, two distinct clients may request to enqueue themselves.

\begin{lstlisting}[language=JavaScript]
    (function main(){
        client1 = ship.initClient().connect("TAQueue");
        client1.send(enqueue, 
            {student: "Arthur", csid: "cs4321"});

        client2 = ship.initClient().connect("TAQueue");
        client2.send(enqueue, 
            {student: "Paul", csid: "cs9876"});
    })();
\end{lstlisting}

{\bf Updating the server state: } Finally, it is necessary to define which data structures or variables are important for a server, thus the {\ttfamily commitState} function receives a function which is executed every time that a service is successfully requested and executed in that server. Revisiting our {\ttfamily server = ship.initServer("TAQueue")} code snippet, we would add a final function to define how the server would be updated after queueing/dequeuing students.

\begin{lstlisting}[language=JavaScript]
    var queue = [];
    function currentQueue() { return queue; }

    (function main(){
        server = ship.initServer("TAQueue");
        ...
        server.commitState(currentQueue);
    })();
\end{lstlisting}


\subsection{What's happening under the hood?}

When a server is initialized and it starts running, we envision that our API will create a state for that server and that this state is updated after the execution of any received message callback. 

As clients connect to a server, the clients themselves assign their own host names and the server keeps a list of connected clients. Upon connection, the clients start a heart beat routine to monitor the server state. Clients are updated in two scenarios, the first one happens through one acknowledgement in the heart beat cycle while the other happens whenever a client receives an acknowledgement from a service usage. In both cases, the acknowledgement carries the server's current state and clients maintain a replica of that state within their storage memory.

Upon failed messages, one client will start its own cached server with the same name as the one that they were connecting to, but followed by a randomly generated unique suffix. As this new server is started and broadcasted in the network, clients will detect the new server based on the original server prefix and thus, they will have to reach a consensus among themselves on who has the most up to date state for the next leading server. Through this process, they exchange their own states such that they can eventually identify which client had the most up-to-date state previous to failure.



\section{Evaluation}
\label{sec:evaluation}

\section{Timeline}
\label{sec:timeline}

We consider late November as the project final deadline. With that in mind, there are a few key activities that we define as milestones, such that we keep up with the project schedule.

\begin{itemize}
    \item {\bf Oct 25th: } Study and evaluate replication patterns. Build a sample FlyWeb queue application;
    \item {\bf Nov 1st: } Implement the core functionalities of the \APIName{} API.
    \item {\bf Nov 8th: } Continue API implementation. Start writing scripts for evaluation;
    \item {\bf Nov 22nd: } Wrap-up API. Write scripts to analyze and plot data; Start drafting project report.
    \item {\bf Nov 30th: } Supposed project deadline?
\end{itemize}

\bibliographystyle{abbrv}
\bibliography{flyweb_paper}

\end{document}

