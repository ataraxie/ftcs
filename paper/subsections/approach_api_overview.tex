\subsection{API Overview}
\label{sub:approach_api_overview}

Successorships provides a framework for fault-tolerant local area Web apps that we named \textit{Shippy}.
Its API is designed to hide underlying details of state replication and the distribution of client and server roles.
The goal of this design is to spare the app developer the complexity of network behavior and let her focus on the implementation details of the app itself.
The library is shipped as a JavaScript file {\ttfamily shippy.js} to be included in HTML files of the Web app.
When loaded, all functionality is exposed on a JavaScript object {\ttfamily Shippy} that resides as a property on the browser's global {\ttfamily window} object.
This object is the only place of interference with the browser's global namespace to avoid naming collisions with the app's environment.

The API consists of three methods exposed on the {\ttfamily Shippy} object as shown in Listing~\ref{lst:api}.
Using these three methods, Shippy apps (1) describe their operations on the app's replicated state ({\ttfamily Shippy.register}), (2) trigger these operations when required ({\ttfamily Shippy.call}) and (3) listen for events dispatched by the framework ({\ttfamily Shippy.on}).
Again, it is important to highlight that these methods will be called from all participants without knowledge of underlying client or server roles. These characteristics will remain within the library and only the state and other information will be dispatched to the app with events.

\begin{lstlisting}[caption={Successorships API},label={lst:api}]
Shippy.register(serviceName, {
    init: function
    operations: <map: name => function>
});
Shippy.call(operationName:string, operationData:object);
Shippy.on(eventName:string, callback:function);
\end{lstlisting}

Usage of the API is best described with the \textit{QueueApp} example from section~\ref{sec:motivating_example}. The app declares a function {\ttfamily init} that describes how the initial state should be manipulated when the app starts. In the case of our QueueApp the state will be a simple queue array that must be set on the state. Listing~\ref{lst:init} shows how this is achieved.

\begin{lstlisting}[caption={QueueApp init function},label={lst:init}]
let init = function(state) {
    state.queue = [];
};
\end{lstlisting}

The operations of the QueueApp are addition and removal of names to and from the queue. These functions will be called by the current server whenever the respective operations are invoked by clients (using {\ttfamily Shippy.call}). By definition, the operation functions will be called with two arguments: the current {\ttfamily state} object and the  {\ttfamily payload} given to {\ttfamily Shippy.call}. Listing~\ref{lst:operations} shows the operations of the QueueApp. The queue is an array that is treated as a \textit{LIFO set} data structure: items are added only if they do not exist and are removed only if they exist.

\begin{lstlisting}[caption={QueueApp operations},label={lst:operations}]
let operations = {
    add: function(state, payload) {
    //add payload.name to state.queue if not existing
    },
    remove: function(state, payload) {
    //remove payload.name from state.queue if existing
    }
};
\end{lstlisting}

The declared {\ttfamily init} function and {\ttfamily operations} object are then registered at the Shippy framework by calling {\ttfamily Shippy.register}. The first argument will be the name of the service to be advertised in the local network for the app, \textit{QueueApp} in our case. Listing~\ref{lst:register} shows the implementation.

\begin{lstlisting}[caption={Call to Shippy.register},label={lst:register}]
Shippy.register("QueueApp", {
    init: init,
    operations: operations
});
\end{lstlisting}