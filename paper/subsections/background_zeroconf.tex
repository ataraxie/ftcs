\subsection{Zeroconf}
\label{sub:background_zeroconf}

Hosts connected to a network rely on maintaining a common consensus on such basic matters as addressing and name resolution, if they are to have any hope of communicating over that network.
This need is often fulfilled by specially-designated configuration servers, such as DHCP or DNS servers.
However, such servers can be absent in common networking situations, such as \textit{ad hoc} networks.
In such cases, it remains desirable for hosts to handle these matters in a seamless and cross-platform fashion, without the need for manual configuration.
This is the problem addressed by the group of protocols referred to as Zeroconf, short for ``Zero Configuration Networking''.\footnote{Although the term is sometimes used more broadly to designate any suite of technologies that seeks to address this problem, we observe the more specific meaning established by the former IETF Working Group by the same name.
See e.g. \url{http://www.zeroconf.org/}.}
Zeroconf protocols solve the problems of address allocation, address queries (mDNS), and network service discovery (DNS-SD), specifically when hosts share a direct link (be it logical or physical).
Since the first of these (address allocation) is a \textit{de facto} standard, as it has been integrated into consumer operating systems and printers since the early 2000s, we focus on the latter of these two here.

\textbf{Address queries.}
A name server can be used on a local network to maintain a mapping of logical names to IP addresses.
In the absence of such a server, a Zeroconf protocol called mDNS (short for ``multicast DNS'') can be used to formulate address queries on the local network.
As per its name, mDNS broadcasts these queries onto the local network rather than directing them to a single server; it also includes a mechanism to resolve naming conflicts~\cite{rfc6762}.
mDNS aims to handle any type of record lookup that could be handled by a DNS server, not just name-to-address lookups.
mDNS has grown steadily in popularity amongst networked devices; implementations of mDNS include Apple Bonjour, Spotify Connect, Philips Hue, Google Chromecast, and Avahi (an open-source implementation for Linux).

\textbf{Service discovery.}
Once connected to a network, a host may wish to learn about the services offered by the other devices connected to the network, such as printing or media streaming.
In a centrally-administered network, this may be accomplished by a directory server, such as a DNS server.
However, when such a server is lacking, the Zeroconf protocol called DNS-SD (``DNS service discovery'') leverages the ability to make distributed DNS queries provided by mDNS, to register, enumerate, and resolve local network services~\cite{rfc6763}.
