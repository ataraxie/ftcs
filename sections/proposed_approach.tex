\section{Proposed Approach}
\label{sec:approach}

Our goal is to build a framework to facilitate the development of offline client-server Web applications that robustly recover from server faults. 
We posit that the following features are prerequisites to achieving this:

\begin{enumerate}
    \item Any client, but exactly one client, has the ability to automatically assume the responsibilities of the server if the server goes down;
    \item All clients in the network have the capability to automatically update their connections to the new server in the event that the server migrates from one node to another;
    
\end{enumerate}

The model we propose for achieving this is one in which clients connecting to the server automatically acquire distributed state including the following elements:

\begin{enumerate}
	\item Constant: A UUID for the initial host node (the first to serve the application);
    \item The current state of the server;
	\item A ``successorship'' list: a list of (potentially not all of the) nodes in the local network, in order of ``who is next'' to assume server responsibilities, in the event that the server goes down;
    \item Constant: The actual server code to execute, in the event that one of the clients needs to begin acting as the server.
\end{enumerate}

Note that the elements marked ``(Constant)'' are permanently fixed (for the lifetime of the application) when the initial server node first runs the application server.

We propose to develop a Javascript library that implements the functionality listed above, providing a clean interface to enable developers to seamlessly integrate fault-tolerance into their offline client-server Web applications, without having to worry about the details of how such fault-tolerance is achieved.

\subsection{API Overview}

Our current running name for the library is \texttt{\APIName}\footnote{We still need a proper acronym for it, e.g. \APIshort}, i.e. the next ship that will lead the flotilla after yet another sunk boat. We propose to implement the following interface in \APIName:


\begin{itemize}
	\item Server side:
    \begin{itemize}
    \begin{ttfamily}
      \item initServer(name)
      \item onReceive(msg, callback)
      \item commitState(callback)
    \end{ttfamily}
    \end{itemize}
    \item Client side:
    	\begin{itemize}
    	\begin{ttfamily}
    		\item initClient()
            \item connect(serverName)
            \item send(msg, payload)
    	\end{ttfamily}
    	\end{itemize}
\end{itemize}


We briefly discuss the major functions of our proposed API in the following subsections. Throughout the discussion, we use the TA queue example to illustrate our API usage.

{\bf Server initialization and service instantiation: } the first functions to initiate a server are {\ttfamily initServer} and {\ttfamily onReceive}. The former starts the local server in the device's browser and, after initialization, assigns a host name for that server. The later register entry points for services offered by that server.

In our queue system, one would initialize a server and define two functions to handle requests to enqueue students and also to dequeue them once they are helped. Additionally, the server provides the queue service in order to provide the current state of the queue. If no recognizable service is requested, the TAQueue server responds with the queue service.

\begin{lstlisting}[language=JavaScript]
    function getQueue(req, event) { ... }
    function handleEnqueue(req, event) { ... }
    function handleDequeue(req, event) { ... }

    (function main(){
        server = ship.initServer("TAQueue");
        server.onReceive('queue', getQueue, 
            default=true);
        server.onReceive('enqueue', handleEnqueue);
        server.onReceive('dequeue', handleDequeue);
    })();
\end{lstlisting}

{\bf Establishing connections: } as a server starts running, it broadcasts its name in the local-area network and clients in the same network can discover this server. A client device needs a single line of code to initialize itself. Upon initialization it will lookup for host servers in the network. Once a list of servers is retrieved and displayed, a client may select a server to connect to. In our TA queue example, we explicitly know the server name and skip the server list phase:

\begin{lstlisting}[language=JavaScript]
    (function main(){
        client = ship.initClient().connect("TAQueue");
    })();
\end{lstlisting}


{\bf Data exchange: } clients can ask for services through the {\ttfamily send} function. The function explicitly takes a requested service as one of its parameters and a payload as its second one. In our queue system, two distinct clients may request to enqueue themselves.

\begin{lstlisting}[language=JavaScript]
    (function main(){
        client1 = ship.initClient().connect("TAQueue");
        client1.send(enqueue, 
            {student: "Arthur", csid: "cs4321"});

        client2 = ship.initClient().connect("TAQueue");
        client2.send(enqueue, 
            {student: "Paul", csid: "cs9876"});
    })();
\end{lstlisting}

{\bf Updating the server state: } Finally, it is necessary to define which data structures or variables are important for a server, thus the {\ttfamily commitState} function receives a function which is executed every time that a service is successfully requested and executed in that server. Revisiting our {\ttfamily server = ship.initServer("TAQueue")} code snippet, we would add a final function to define how the server would be updated after queueing/dequeuing students.

\begin{lstlisting}[language=JavaScript]
    var queue = [];
    function currentQueue() { return queue; }

    (function main(){
        server = ship.initServer("TAQueue");
        ...
        server.commitState(currentQueue);
    })();
\end{lstlisting}


\subsection{Replication Strategy}

We aim to leverage the concepts of \textbf{eventual consistency} and \textbf{lazy replication} for our fault tolerance approach. 
We argue that our domain of local ad-hoc offline Web applications will mostly tolerate minor time frames of inconsistent states and we rather aim to increase availability and performance.

When a server is initialized and starts running, we envision that our API will create a state for that server and that this state is updated after the execution of any received message callback. 
As clients connect to a server, the clients themselves assign their own host names and the server keeps a list of connected clients.
Changes to the server state are broadcasted to all clients. 
A central concern, therefore, is to handle the possibility of inconsistencies in the versions of the server state maintained by each client: what if the server dies before changes are broadcasted? 
What if the broadcast to our next server fails and the next server takes over with a stale state? 

Our current idea is to converge to a consistent state in accordance with the eventual consistency principle when clients (re-)connect to the new server: upon connecting, they can send their current state associated with a timestamp of the last modification.
From this information, the most current version of the previous server's state can be determined from amongst all clients' copies, and then propagated to the new server.
This system opens up the risk of being spoofed by malicious clients imposing a manipulated state onto our network, but as stated previously, we allow ourselves to assume full trust between all participants in the network for simplification.
 
We have also considered using a CRDT for our implementation, but we believe that the constraint of commutativity of operations, or associativity of merging conflicting states, is potentially too restricting for the range of applications we would want to allow to run on our framework.


\subsection{Assumptions}

Our first version of this fault-tolerant extension to zero-configuration networks will necessarily have to make some simplifying assumptions, to make initial design and implementation tractable. 
In particular, we assume the following, fully realizing that these may not hold for real-world practical use cases:

\subsubsection*{The nodes in the network trust one another.} 

An implication of our model is that any node in the network (with a reliable connection) may at some point become the application server.
This status comes with all the responsibilities of the server, including executing the server-side application code, and hence maintaining the server-side application state.
Without any further substantial design considerations, this model would be extremely susceptible to a malicious client acquiring server status, and exploiting this status for their own ends, either by serving downright malicious material, or by more subtly forging application state.
For an example of the latter, consider once again the \texttt{queue} app; when the TA leaves the room and one of the student devices becomes the server, that student could leverage her trusted position as the new server by pushing herself to the front of the line.

% TODO (PTC): State further assumptions here.

