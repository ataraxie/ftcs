\subsection{Related Platforms}
\label{sec:related_platforms}

The concept of creating local ad-hoc networks with very little configuration has been present for several years.
The first implementation that tried to bring the concept to Web developers came from \textit{Opera Labs} with \textit{Opera Unite}\footnote{http://www.operasoftware.com/press/releases/general/opera-unite-reinvents-the-web}.
Announced in 2009, this implementation certainly did not receive appropriate attention and was archived soon after publishing.
More recently in 2016, \textit{Google} presented their \textit{Nearby} project that enables Android devices to communicate with other closely located devices forming a peer-to-peer network.
Initial service discovery is achieved through bluetooth.
Obviously, Google's solution is mostly based on their own Android platform which basically makes it unpractical in the common scenario of participants with different operating systems.

\textit{FlyWeb} is a Web API developed by the Mozilla Firefox community which enables clients of Web applications to publish a local server from within the browser.
Building on the concept of zero-configuration networks and its mDNS/DNS-SD protocols~\cite{rfc6762, rfc6763}, the server advertises itself in the local network and can be discovered by other devices which become clients to the server by connecting via a HTTP or WebSocket connection.
This essentially enables cross-device communication within a local-area network.

For our approach, we decided for FlyWeb since it is open source and entirely based on Web technology which makes it widely available.
We think that the idea of hosting an ad-hoc network from within a Web application in the browser has great potential and we think that applications for this network can benefit from graceful recovery upon disconnection of local servers.