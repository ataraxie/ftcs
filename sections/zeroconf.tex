\subsection{Zero-configuration Networks}
\label{sec:zeroconf}



Zero-configuration networking is a combination of protocols that aims to automatically discover computers or peripherals in a network without any central servers or human administration. To do so, Zero-configuration networks have two major components that provide {\it (i)} the automatic assignment of IP addresses and host naming (mDNS), and {\it (ii)} service discovery (DNS-SD).


Roughly, when a device enters the local network, it assigns an IP/name to itself and then multicasts that to the local network, resolving any name conflicts that may occur in the process. IP assignment considers the link-local domain address, which draws addresses from the IPv4 169.254/16 prefix and, once an IP address is selected, a host name with the suffix ``.local'' is mapped to that IP~\cite{rfc6762}. As devices are mapped to IPs/host names, their available services are discovered using a a combination of DNS PTR, SRV, and TXT records~\cite{rfc6763}; thus, their services can be requested by other devices. 

Despite smoothly assigning names and discovering services, zero-configuration networks do not address client disconnection. As a consequence, when a device disconnects from the network, communication to that device ends abruptly. 


